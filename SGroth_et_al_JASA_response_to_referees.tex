\documentclass[11pt]{article}
\usepackage{pgf,tikz}
\usetikzlibrary{arrows}
\usepackage{amssymb}
\usepackage{parskip}
\usepackage[margin=2cm]{geometry}



%--------------------------------------------------------------------------------------------------------------
\begin{document}
\title{Response to reviewers' comments on JASA-06265 \\
``Accelerating numerical methods for nonlinear acoustics using nested meshes''}
\author{S.~P.~Groth, P.~G\'elat, S.~R.~Haqshenas, N.~Saffari, E.~van `t Wout, T.~Betcke, G.~N.~Wells}
\maketitle

Dear Dr.~Gumerov,

Thank you for handling our submission to JASA. We are grateful to the reviewers 
for their helpful and stimulating comments. 
We have outlined our responses below - we hope this is all acceptable to you. 
Please find attached both a revised manuscript with the revisions highlighted 
in red and the same manuscript with the red font replaced with black. 

Kind regards,\\
S.~P.~Groth, P.~G\'elat, S.~R.~Haqshenas, N.~Saffari, E.~van `t Wout, T.~Betcke, G.~N.~Wells


\section*{Reviewer 1}
\begin{enumerate}
	\item \textbf{Comment:} Unecessary Re\{\} -operator in eq. (3). Or perhaps the authors meant to have eq. (3) in the form without the adjoints.\\
	\textbf{Response:} We have changed equation (3) to be in the form without the adjoints,
	which was the original intention.
	
	\item \textbf{Comment:} Add a comment after eq. (10) for situation $x = y$.\\
	\textbf{Response:} We have added a sentence after eq. (10) discussing the singularity that occurs at $x=y$.
	
	\item \textbf{Comment:} Eq. (13), p2(y), y is not in bold.\\
	\textbf{Response:} y is now bold.
	
	\item \textbf{Comment:} P. 11, ``In a homogeneous medium, the first harmonic 
	p1 is merely the incident field generated by the transducer, which we can 
	compute anywhere in R3.'' -- Should specify how this is done. Or point out 
	this will be discussed later.\\
	\textbf{Response:} Have added a phrase pointing out that this is discussed 
	later, in Section III B.
	
	\item \textbf{Comment:} P. 12, ``The Jacobian of the transformation to spherical coordinates 
	cancels the singularity in the Green's function.'' -- The authors mean that 
	when the coordinate transformation is made for the integral, the ensuing 
	Jacobian determinant cancels the singularity? Should rephrase for clarity. \\
	\textbf{Response:} Added the following sentence to clarify:\\
	``This integral can then be transformed into spherical coordinates, 
which is convenient because the Jacobian determinant of the transformation cancels
the singularity in the Green's function.''
	
\item \textbf{Comment:} Might want to consider using e.g. LaTeX \verb!\imath!
	 or \verb!\jmath! for the imaginary symbol in the manuscript just as not to 
	 confuse with i and j indices. This is a small issue in e.g. eq. (17). \\
	 \textbf{Response:} Have modified imaginary $i$ to be represented in Roman 
	 script i, rather than italic, to differentiate from indices.
	 
	 \item \textbf{Comment:} Line 216, what is the number of points per wavelength 
	 for choice of 4096 discretization points? \\
	 \textbf{Response:} Have added `which equates to approximately 
	 ten monopole sources per fundamental wavelength' to clarify this.
	
	 \item \textbf{Comment:} Table 1: Is the small r (inner radius) necessary? \\
	\textbf{Response:} We agree. Have omitted the small r from Table I and Figure 1.
	
	\item \textbf{Comment:} Lines 237-240: Unclear what singularity and why epsilon is used? \\
	\textbf{Response:} Have added some explanation on lines 239-240 to explain 
	that the epsilon is a shift required to avoid the possibility of evaluating 
	the monopole sources (Green's functions) at locations for which they are undefined, 
	i.e., on the tranducer bowl's surface: $x=r_i$.
	
	\item \textbf{Comment:} Is equation on line 237 correct for L? For transudcer H101 this would imply value of -0.1+j10.1, and for H131 value of 11.6. \\
	\textbf{Response:} Yes, the equation is correct. However, the figures for R 
	in Table 1 were incorrect, leading to the bizarre values for $L$ derived by the reviewer. They were previously for the diameter by mistake.
	We have corrected the figures for R in Table 1.
	
	\item \textbf{Comment:} Should add explanation after (21) why this $L$ parameter is defined at all. Likewise for $d$. \\
	\textbf{Response:} We define $L$ since it represents the length of the default computation domain,
	which we later aim to shrink to save on computational load. After (21) we have added some 
	further discussion of length and width of the computation domain to make it more apparent 
	why we are defining them here and why these definitions will be of use later in the paper.
	
	\item \textbf{Comment:} Section IV. Why was HITU Simulator used? Why no comparison with lets say k-Wave? Or analytical solutions. \\
	\textbf{Response:} HITU Simulator was used since it is a well-established computational tool in the 
	computational acoustics community and is well suited and efficient for the axisymmetric configurations considered in this paper. 
	Regarding analytical solutions, the first harmonic computed using our proposed technique is in fact an evaluation 
	of the analytical solution for the linear problem. However, analytical solutions are not available for the nonlinear setting with focused transducers.

	\item \textbf{Comment:} Lines 338-342: What is exactly "fraction of the computation domain"? \\
	\textbf{Response:} We have added a sentence at line 357 to make clearer what we mean by this phrase.

	\item \textbf{Comment:} Since eq. (27) is not utilized at all why is it presented? \\
	\textbf{Response:} Eq.~(27) is presented since we believe that it is an interesting observation that the error decays in proportion to the 
	square root of the field's magnitude. Although we have not made strict use of this observation in the article,
	we hope that a future reader might be able to either explain or make use of this observation to further optimise the efficiency of similar computational methods
	to that presented in this paper. 

	\item \textbf{Comment:} Lines 360, 364 use micrometers. \\
	\textbf{Response:} Have amended to use micrometers.
	
	\item \textbf{Comment:} Lines 359-368 add number of discretization points for each harmonic with the nested approach and their sum. \\
	\textbf{Response:} Have added the requested information.

	\item \textbf{Comment:} Discuss how the ``rule of thumb'' for choosing fraction of simulation volume for higher harmonics generalizes to other transducer configurations and other frequencies. \\
	\textbf{Response:} We have added a paragraph to the end of the conclusion providing such a discussion.
	
	\item \textbf{Comment:} A good test for the rule of thumb is see how it works for a piston transducer since the 
	rule of thumb was derived based on focused transducers. 
	If it does not work then discussion should be added on its limits.\\
	\textbf{Response:} We have chosen to concentrate on focused ultrasound fields in 
	this article, which form an important set of problems. In order to make this specialisation 
	more apparent, we have made the title more specific as well as the discussion in 
	the introduction. Therefore, we do not consider the piston transducer problem in 
	this article.
	
	\item \textbf{Comment:} The discussion on the approach being usable with finite element methods is vague. Clarify if you mean that computation of p1 could be replaced by any model. \\
	\textbf{Response:} We agree. The mention of finite element methods is not necessary. The discussion is intended to 
	refer to more general numerical methods for both the full Westervelt equation, which 
	would typically employ one single mesh, and for methods using a separate mesh for each 
	harmonic, as was done in the paper. We are claiming that the nested meshing rule of thumb 
	devised can also be used in the single mesh methods to dictate efficient refining of the mesh towards the focus.
	We have modified the first two sentences of the conclusion to make this point clearer.
	
	\item \textbf{Comment:} Lines 465-467: The authors trivialize extension of the model to inhomogeneous or scattering setup far too much. Consider omitting, or extending. \\
	\textbf{Response:} We have added Section VII to discuss the extension to inhomogeneous
	media and provide a comparison with HITU for a simple setup with a layer of differing material. This 
	provides a good agreement, thus suggesting that our approach generalises well to at least 
	simple inhomogeneous problems.

	\item \textbf{Comment:} It might be useful to have a figure of memory consumption and computational time as a function of Q0. \\
	\textbf{Response:} We respectfully disagree that such a figure would be of value to the discussion in Section V. This is 
	because our aim is to understand the minimal size of the domain required to capture all the important 
	contributions to each particular harmonic. Then the computational savings are a consequence of the 
	reduction in domain size. Whereas the suggested figure implies that we are striving to achieve a given 
	computational load, without necessary much concern for accuracy.
\end{enumerate}

\section*{Reviewer 2}
\begin{enumerate}
	\item \textbf{Comment:} The approach uses a method of successive approximations by 
	calculating the field of each higher harmonic using the fields of the previous ones. Is it assumed therefore that the field 
	of the first harmonic does not change due to nonlinear propagation effects. The algorithm thus does not account for backward transfer of the energy from higher harmonics to the lower ones (Eqs. 12 - 16), i.e. it is valid only for weakly nonlinear fields.
	If such fields are of interest, they are not very related to strongly focused HIFU, as for HIFU nonlinear effects become of importance only when they are strong (see detail comments below).
-       For weakly nonlinear less focused diagnostic sources such approach may help but still neglecting accurate governing of nonlinear cascade processes is a significant drawback of the method.\\
	\textbf{Response:} We agree and indeed it is this weakly nonlinear case that we are focusing on
	in this article. Therefore, we have made this more explicit throughout the article, including 
	making the title more specific to the weakly nonlinear problem.
	
	\item \textbf{Comment:} Most interesting would be to demonstrate a possibility of this approach for full-wave propagation, at least introducing a boundary and obtain nonlinear reflection and the wave propagating backward. Is this possible? \\
	\textbf{Response:} We have added Section VII in which we generalise to approach to 
	inhomogeneous problems. There with perform a comparison with HIFU for a layer of kidney material immersed in water.
	
	\item \textbf{Comment:} Very similar optimizations (and many other) of nonlinear HIFU beam simulation have been explored previously and are already employed in modeling ultrasound fields generated by therapeutic focused sources of various geometry. It seems valuable to compare the proposed optimizations to the existing ones.
	For example, for one-way simulation of the Westervelt equation, the approach has been introduced to extend the wave spectrum and reduce a spatial window in 3D simulations when propagating to the focus. Such approach has employed for modelling even extreme cases of shock formation (1). This way, both nonlinear distortion of the acoustic waveform and the geometry of the field are optimized. Simulations have shown very good agreement with measurements (2,3) thus validating such optimization. \\
	1 P.V. Yuldashev, V.A. Khokhlova. Simulation of three-dimensional nonlinear fields of ultrasound therapeutic arrays. Acoustical Physics, 2011, v. 57(3), pp. 334-343.\\
	2 W. Kreider, P.V. Yuldashev, O.A. Sapozhnikov, N. Farr, A. Partanen, M.R. Bailey, and V.A. Khokhlova. Characterization of a multi-element clinical HIFU system using acoustic holography and nonlinear modeling. IEEE Trans. Ultrason., Ferroelect., Freq. Contr., 2013, v. 60(8), pp.1683-1698.\\
	3 M.M. Karzova, P.V. Yuldashev, O.A. Sapozhnikov, V.A. Khokhlova, B.W. Cunitz, W. Kreider, M.R. Bailey. Shock formation and nonlinear saturation effects in the ultrasound field of a diagnostic curvilinear probe, J. Acoust. Soc. Am., 2017, v. 141(4), pp. 2327-2337.\\
	The authors of the paper propose to additionally reduce the size of the grid step for higher harmonics, which could add more optimization. It would be helpful to compare and correlate the novelty of the present effort with the existing validated approach.  \\
	\textbf{Response:} We thank the reviewer for these recommendations. Indeed, the approach employed in these methods 
	possesses similarities to our approach and we have added a paragraph in the introduction (pp.4-5) discussing the 
	similarities and differences between this prior work and our work. The main differences are that 
	we propose modifying the grid step for each harmonic, thus leading to further computational savings, and 
	also we provide a through study of how the domain sizes for each harmonic are chosen, whereas in 
	these previous works, no details of why the particular domain sizes are chosen are given.

	\item \textbf{Comment:} P5, L25 Tissue death can be caused directly via thermal  ablation, or via other mechanisms such as cavitation.\\
	Two mechanisms are directly related to nonlinear wave propagation such as shock-scattering histotripsy (4) and boiling histotripsy (5) and worth mentioning: \\
	(4) A.D. Maxwell, T.Y. Wang, C.A. Cain, J.B. Fowlkes, O.A. Sapozhnikov, M.R. Bailey, and Z. Xu "Cavitation
	clouds created by shock scattering from bubbles during histotripsy," J. Acoust. Soc. Am. vol. 130, no 4, pp. 1888-98, 2011.\\
	(5) T.D. Khokhlova, M.S. Canney, V.A. Khokhlova, O.A. Sapozhnikov, L.A. Crum, M. R. Bailey. Controlled tissue emulsification produced by high intensity focused ultrasound shock waves and millisecond boiling. J. Acoust. Soc. Am., 2011, v.130(5), pp. 3498-3510.	
	
	\textbf{Response:} We have added mentions of these two mechanisms and the suggested references.
		
	\item \textbf{Comment:} P5, L26 In the thermal ablation setting, which is the focus of this article, the peak acoustic pressure is often between 1 and 10 MPa, at which linear acoustic theory is no longer accurate.
	It has been well accepted, that nonlinear propagation effects do not noticeably effect the volume of thermally ablated tissue unless the propagation is highly nonlinear and shock-wave heating leads to generation of boiling bubbles at the focus. The paper referenced (Solovchuk 31 et al., 2014) actually shows that the difference in temperature between linear and nonlinear not very nonlinear simulations is present only at the focus and negligible at the borders that define thermally ablated focal region (Fig. 2). Taking this into account, the most important is to introduce optimization for the cases when many harmonics are included, at least hundreds. For full-wave modeling, this would be an important contribution. But this would not work with the proposed method. \\
	\textbf{Response:} We agree that the difference in temperature between linear and 
	weakly nonlinear simulations is present only near the focus, and indeed this the motivation for the domain mesh 
	optimisation performed in this work. We also agree that a significant contribution would 
	be optimisation for problems containing hundreds of harmonics, however that is a 
	different and more complicated setting than the one we consider in this paper. As mentioned in responses to 
	prior comments, we have rephrased the title and discussions throughout the paper to highlight that we are 
	here concentrating on the weakly nonlinear case. 
	
	\item \textbf{Comment:} P5, L32 Furthermore, the location of the focus in the nonlinear regime is often shifted relative to that predicted by the linear theory (Camarena et al., 2013).
	This paper (Camarena et al., 2013) is related to the weakly focused transducers, not HIFU. In HIFU beams, indeed, the focus (position of the peak positive pressure) shifts with the increase of the peak power, but this shift is small. Not obvious how this effect is related to the study. \\
	\textbf{Response:} We agree that this effect is not related to this study so have deleted this sentence to avoid any confusion.

	\item \textbf{Comment:} P5, L41 About efforts on optimization nonlinear schemes
	In addition to earlier comments, very effective optimization is to combine shock-capturing Godunov-like schemes with the frequency-domain ones when strong distortion of the waveform is achieved (1, 2, 3). Axially-symmetric beams have been optimized this way using both parabolic and wide-angle parabolic approximations without using clusters:\\
	6. O.V. Bessonova, V.A. Khokhlova, M.R. Bailey, M.S. Canney, and L.A. Crum. Focusing of high power ultrasound beams and limiting values of shock wave parameters, Acoust. Phys., 2009, Vol. 55, No. 4-5, pp. 463-473.\\
	7. P.V. Yuldashev, I.S. Mezdrokhin, V.A. Khokhlova. Wide-angle parabolic approximation for modeling high-intensity fields from strongly focused ultrasound transducers. Acoust. Phys., 2018, v. 64(3), с. 309-319.\\
	\textbf{Response:} \textcolor{red}{Help}

	\item \textbf{Comment:} P7, L41 About reducing full-wave Westervelt equation to the Hemholts equations (Soneson 2017)
	This is not exactly correct. Finally, the codes developed by J. Soneson (refered in the paper) and the approach indicated earlier here (6, 7), deal with solving one-way version of the Westervelt equation in the retarted coordinates thus excluding the dispersion problem of transferring the wave numerically over many wavelengths. \\
	\textbf{Response:} We agree that the reference to the Soneson paper as being similar 
	to our reduction to Helmholtz equations is not correct, since Soneson does not neglect the transfer of 
	energy from higher to low harmonics, as we do. So we can amended to the citation, now referring only to (Du and Jensen, 2013), who do make the same assumptions as we do.

	\item \textbf{Comment:} P8, L95  Considering a weakly nonlinear case for one-way propagation does not make too much sense as this does not represent any important HIFU problem. Moreover, already many codes can efficiently do this (Soneson, Yuldashev). P9, L41 For implementation of the algorithm for solving a 3D full-wave equation. \\
	\textbf{Response:} We have added Secton VII to generalise our approach to the inhomogeneous medium 
	case in which wave scattering is observed. These problems are more relevant to practical focused ultrasound.

	\item \textbf{Comment:} P9, L21 About the power law of absorption
	If the power law of absorption is introduced, a corresponding dispersion should be accounted for. It is missed here. Was dispersion included in simulations? The most accurate representation can be found in \\
	K. R. Waters, J. Mobley, and J. G. Miller, "Causality-imposed (Kramers-Kronig) relationships between attenuation and dispersion," IEEE Trans. Ultrason., Ferroelect., Freq. Control, vol. 52, no. 5, pp. 822-823, 2005." \\
	\textbf{Response:} \textcolor{red}{Help}

	\item \textbf{Comment:} P10, L41. Comparing time- and frequency-domain approaches. 
	A combination of both, when frequency domain is used for slightly nonlinear propagation turning to the time-domain with shock-capturing scheme when acoustic waveform has been developed earier by others. \\
	\textbf{Response:} \textcolor{red}{Help}
	
	\item \textbf{Comment:} P11, L157 About weak or strong nonlinearities, 20 MPa for histotripsy
	Weakly nonlinear cases are relevant to some HIFU, but do not affect treatments much.
	Pressure levels for histotripsy are typically higher than 80 MPa for the peak positive pressure.\\
	\textbf{Response:} \textcolor{red}{Help}

	\item \textbf{Comment:} P15, L200  About modeling nonlinear fields from multi-element phased arrays. 
	The paper (2) could represent a good example of applying optimized modeling (with reduced spatial window for harmonics and combined frequency and time domains) with experimentally obtained boundary condition.\\
	\textbf{Response:} Have added the suggested reference.

	\item \textbf{Comment:} P19, L270 Power of 50W and 100W in simulations. \\
	Such power outputs are not relevant for HIFU when thermal tissue ablation is considered. \\
	\textbf{Response:} \textcolor{red}{Help. Is this not a problem now we've switched to FUS?}
 
	\item \textbf{Comment:} P22, L282 We observe that the volume potential method predicts a slightly larger amplitude
	This is because only the energy flow from lower harmonics towards the higher ones is considered.\\
	\textbf{Response:} We agree that might indeed be the case and hence have inserted a sentence
	proposing this as a possible explanation.
	
	\item \textbf{Comment:} Transducer parameters are confusing F-number, according to the table is 0.5, not 1 as shown in Fig. 1.
	Is it the diameter or radius of the transducers indicated in the table? \\
	\textbf{Response:} This is due to an error in Table I, where the diameter instead 
	of the radius was given. This can been corrected.
\end{enumerate}


\end{document}